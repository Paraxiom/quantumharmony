\documentclass[11pt]{article}

\usepackage[a4paper,margin=1in]{geometry}
\usepackage{longtable}
\usepackage{hyperref}
\usepackage{booktabs}
\usepackage{array}
\usepackage{setspace}
\usepackage{titlesec}

\titleformat{\section}{\large\bfseries}{\thesection}{1em}{}
\titleformat{\subsection}{\normalsize\bfseries}{\thesubsection}{1em}{}

\setstretch{1.15}

\title{\textbf{QuantumHarmony Light Paper}\\
\vspace{0.3em}
\large Version 1.5 — January 2025}

\author{QuantumVerse Protocols}
\date{}

\begin{document}
\maketitle

\begin{quote}
\textbf{Changelog v1.5}: Added Web Interface section. Added Research Publications with DOIs (6 papers on Zenodo). Expanded references.

\textbf{v1.4}: Added Quantum P2P networking section (ML-KEM-1024, Falcon-1024, QKD hardware interface). Expanded Proof of Coherence with P2P integration details.

\textbf{v1.3}: Added Use Cases section (QCAD, Fideicommis, Pedersen). Expanded governance documentation. Added Triple Ratchet encryption and 512-segment toroidal mesh documentation.

\textbf{v1.2}: Added MEV protection documentation. Corrected finality description—QuantumHarmony provides deterministic BFT finality via the Coherence Gadget, not probabilistic finality.
\end{quote}

\section*{Abstract}

QuantumHarmony is a Layer 1 blockchain built on Substrate that replaces quantum-vulnerable cryptographic components with post-quantum alternatives. This document describes what the system does, how it works, and its current state.

\section{Problem Statement}

\subsection{Quantum Computing Threat}

Current blockchains rely on cryptographic primitives that quantum computers can break:

\begin{center}
\begin{longtable}{@{}lll@{}}
\toprule
\textbf{Primitive} & \textbf{Algorithm} & \textbf{Quantum Attack} \\
\midrule
Signatures & ECDSA, Ed25519 & Shor's Algorithm \\
Finality & BLS (GRANDPA) & Shor's Algorithm \\
Hashing & Blake2b & Grover's Algorithm \\
\bottomrule
\end{longtable}
\end{center}

\noindent
\textbf{Fact}: NIST estimates cryptographically relevant quantum computers could exist within 10–15 years. Blockchain addresses and signed transactions recorded today become vulnerable once such computers exist.

\subsection{What QuantumHarmony Changes}

\begin{center}
\begin{longtable}{@{}lll@{}}
\toprule
\textbf{Component} & \textbf{Standard Substrate} & \textbf{QuantumHarmony} \\
\midrule
Signatures & Ed25519 / ECDSA & SPHINCS+ (NIST PQC) \\
Block Hashing & Blake2b & Keccak-256 (SHA-3) \\
Finality Gadget & GRANDPA (BLS) & Coherence Gadget (Falcon1024) \\
Randomness & VRF & Quantum-enhanced VRF (optional QKD) \\
\bottomrule
\end{longtable}
\end{center}

\section{Technical Implementation}

\subsection{SPHINCS+ Signatures}

SPHINCS+ is a stateless hash-based signature scheme standardized by NIST in 2024. Its security relies solely on hash function properties, not discrete logarithms or elliptic curves.

\textbf{Trade-offs}:
\begin{itemize}
  \item Signature size: approximately 8–50 KB
  \item Slower signing compared to Ed25519
  \item Verification time comparable to classical schemes
\end{itemize}

Implementation is provided via \texttt{pallet-sphincs-keystore}.

\subsection{Keccak-256 Hashing}

Keccak-256 (SHA-3) replaces Blake2 throughout the runtime.

\begin{itemize}
  \item 256-bit output provides 128-bit post-Grover security
  \item 1600-bit sponge state
  \item Standardized and widely audited
\end{itemize}

\subsection{Triple Ratchet Encryption}

Validator-to-validator communication uses a \textbf{Triple Ratchet} protocol combining three key rotation mechanisms:

\begin{enumerate}
  \item \textbf{Falcon Ratchet}: Long-term post-quantum signatures (slow rotation)
  \item \textbf{Merkle Ratchet}: Hierarchical key derivation (periodic rotation)
  \item \textbf{Symmetric Ratchet}: Ephemeral session keys (per-message rotation)
\end{enumerate}

This provides forward secrecy: compromise of current keys does not reveal past messages.

\subsection{512-Segment Toroidal Mesh}

Runtime execution is parallelized across an 8×8×8 toroidal mesh (512 segments):

\begin{itemize}
  \item Each segment handles a subset of accounts
  \item Parallel transaction execution within segments
  \item Cross-segment communication via 6 neighbors (3D torus)
  \item Load balancing with automatic rebalancing
  \item Maximum 3 hops between any two segments
\end{itemize}

Implementation: \texttt{pallet-runtime-segmentation}

\subsection{Quantum-Secured P2P Networking}

Validator-to-validator communication uses a fully post-quantum secured P2P layer:

\textbf{Identity \& Key Exchange}:
\begin{itemize}
  \item \textbf{Falcon-1024}: Node identity and message signing (NIST PQC)
  \item \textbf{ML-KEM-1024} (Kyber): Key encapsulation for session establishment (NIST PQC)
  \item \textbf{AES-256-GCM}: Authenticated encryption for message confidentiality
\end{itemize}

\textbf{Protocol flow}:
\begin{enumerate}
  \item Node generates Falcon-1024 signing keypair + ML-KEM-1024 KEM keypair
  \item Session initiation: ML-KEM encapsulation creates shared secret
  \item Shared secret derives AES-256 session key
  \item All messages signed with Falcon-1024, encrypted with AES-256-GCM
  \item Automatic key rotation (default: 1 hour)
\end{enumerate}

\textbf{QKD Hardware Integration}: When QKD hardware is available, session keys can be derived from QKD-generated entropy instead of ML-KEM. Supported vendors (stubs ready): Toshiba, ID Quantique, QuantumCTek, SK Telecom, NTT. Interface follows ETSI GS QKD 014.

\subsection{Consensus and Finality}

\textbf{Block production}: Aura (Authority Round).

\textbf{Finality}: Deterministic BFT finality via the \textbf{Coherence Gadget}, a post-quantum replacement for GRANDPA.

\subsection{Coherence Gadget}

The Coherence Gadget provides GRANDPA-equivalent deterministic finality:

\begin{center}
\begin{longtable}{@{}ll@{}}
\toprule
\textbf{GRANDPA} & \textbf{Coherence Gadget} \\
\midrule
BLS signatures & Falcon1024 signatures \\
Prevote / Precommit & STARK verification + coherence scoring \\
2/3 supermajority & 2/3 supermajority \\
Finality proof & Finality Certificate \\
\bottomrule
\end{longtable}
\end{center}

\textbf{Protocol flow}:
\begin{enumerate}
  \item New block produced by Aura
  \item Proof collection (entropy / coherence inputs)
  \item Proof verification and scoring
  \item Falcon1024 signing
  \item Vote broadcast (encrypted)
  \item Supermajority aggregation
  \item Finality certificate generation
\end{enumerate}

\subsection{Proof of Coherence (PoC)}

PoC is the consensus mechanism that combines quantum entropy with BFT finality.

\textbf{With quantum hardware}:
\begin{itemize}
  \item QRNG / QKD entropy sources (Toshiba, Crypto4A, IdQuantique)
  \item STARK proofs verified with Winterfell
  \item QBER-based coherence scoring (threshold: 11\%)
  \item QKD-derived session keys for validator P2P
\end{itemize}

\textbf{Without hardware (fallback)}:
\begin{itemize}
  \item Mock entropy sources for testing
  \item ML-KEM-1024 session keys for validator P2P
  \item Full BFT execution preserved
  \item Falcon1024 signatures provide post-quantum security
  \item Deterministic finality still guaranteed
\end{itemize}

\textbf{Integration with P2P Layer}:
\begin{itemize}
  \item Coherence votes broadcast via quantum-secured P2P channels
  \item Vote encryption uses QKD-derived keys when available, ML-KEM otherwise
  \item Triple Ratchet provides forward secrecy for vote messages
\end{itemize}

Quantum hardware improves entropy quality but is not required for correctness or finality.

\subsection{MEV Protection}

QuantumHarmony provides native Maximal Extractable Value (MEV) protection at the protocol level.

\textbf{The problem}: In traditional blockchains, validators can reorder transactions (frontrunning), insert their own transactions (sandwich attacks), or censor specific transactions.

\textbf{Solution}:
\begin{enumerate}
  \item Leader elected via quantum-seeded VRF (unpredictable)
  \item Leader maintains qVRF-ordered priority queue
  \item Leader compares priority queue against public mempool
  \item Discrepant transactions are deleted
\end{enumerate}

\textbf{Reporter requirements}: Every report must include a randomly generated nonce:
\[
\texttt{tx\_hash} = \texttt{Hash}(\texttt{payload} \| \texttt{random\_nonce})
\]

This ensures unique transaction hashes, prevents replay attacks, and enforces deterministic ordering.

\begin{center}
\begin{longtable}{@{}ll@{}}
\toprule
\textbf{Attack} & \textbf{Mitigation} \\
\midrule
Frontrunning & qVRF ordering is unpredictable \\
Sandwich attacks & Discrepancy detection removes injected txs \\
Transaction censorship & Leader rotation \\
Replay attacks & Random nonce per report \\
\bottomrule
\end{longtable}
\end{center}

\section{Use Cases}

\subsection{QCAD Stablecoin}

Canadian dollar stablecoin (\texttt{pallet-stablecoin}):
\begin{itemize}
  \item 1:1 CAD peg via oracle price feeds
  \item Collateralized vaults (150\% minimum ratio)
  \item Liquidation engine for undercollateralized positions
  \item Stability fees paid in native token
\end{itemize}

\subsection{Fideicommis Trusts}

Quebec Civil Code compatible trust administration (\texttt{pallet-fideicommis}):
\begin{itemize}
  \item Trust creation with grantor, trustee, beneficiaries
  \item Asset registration (on-chain and off-chain references)
  \item Distribution rules: time-locked, conditional, discretionary
  \item Trustee succession with multi-sig handoff
\end{itemize}

\subsection{Pedersen Commitments}

Zero-knowledge proofs on BLS12-381 (\texttt{pallet-pedersen-commitment}):
\begin{itemize}
  \item Commit-reveal for MEV protection
  \item Range proofs for private amounts
  \item Binding + hiding properties
\end{itemize}

\section{Governance System}

QuantumHarmony includes standard Substrate governance pallets:
\begin{itemize}
  \item Democracy
  \item Collective
  \item Treasury
  \item Scheduler
\end{itemize}

\subsection{Academic Vouching}

Credential verification system (\texttt{pallet-academic-vouch}):
\begin{itemize}
  \item Institution registration (universities, certification bodies)
  \item Credential issuance with expiry
  \item On-chain voting for academic registration
  \item Vouch threshold for program acceptance
\end{itemize}

\subsection{Ricardian Contracts}

Human + machine readable legal contracts (\texttt{pallet-ricardian-contracts}):
\begin{itemize}
  \item Dual format: legal prose + executable code
  \item Multi-party signing workflow
  \item Amendment tracking with version history
  \item State transitions: Draft $\rightarrow$ Active $\rightarrow$ Executed/Terminated
\end{itemize}

\subsection{Notarial Services}

Document attestation system (\texttt{pallet-notarial}):
\begin{itemize}
  \item Hash attestation with timestamp proof
  \item Witness certification system
  \item Certificate generation
  \item Revocation with reason codes
\end{itemize}

\section{Current State}

\textbf{Testnet}: Operational (3 validators) \\
\textbf{Block time}: 6 seconds \\
\textbf{Consensus}: Aura + Coherence Gadget

\textbf{What works}:
\begin{itemize}
  \item Block production with Aura + SPHINCS+
  \item Deterministic BFT finality via Coherence Gadget
  \item All governance and legal pallets
  \item STARK proof verification path
  \item Docker deployment
\end{itemize}

\textbf{In progress}:
\begin{itemize}
  \item Production QKD hardware integration
  \item Multi-region validator expansion
\end{itemize}

\textbf{Not done}:
\begin{itemize}
  \item Security audit
  \item Mainnet launch
\end{itemize}

\section{Limitations}

\begin{itemize}
  \item Large post-quantum signatures (~29 KB for SPHINCS+, ~1.3 KB for Falcon1024)
  \item No BLS-style signature aggregation
  \item Non-standard Substrate tooling compatibility
\end{itemize}

\section{Comparison}

\begin{center}
\begin{longtable}{@{}llll@{}}
\toprule
 & \textbf{QuantumHarmony} & \textbf{Substrate} & \textbf{QRL} \\
\midrule
Signatures & SPHINCS+ / Falcon & Ed25519 / BLS & XMSS \\
Finality & Deterministic BFT & GRANDPA (BFT) & PoW \\
MEV Protection & Native (qVRF) & No & No \\
Quantum HW & Optional & No & No \\
\bottomrule
\end{longtable}
\end{center}

\section{Web Interface}

A web-based notarial interface is available for end users:

\begin{itemize}
  \item Document attestation with SPHINCS+ signatures
  \item Contract creation and multi-party signing
  \item QCAD stablecoin transactions
  \item Fideicommis trust management
  \item Account creation with local key storage
\end{itemize}

\textbf{Technical stack}: SHA-256 document hashing, SPHINCS+-256s post-quantum signatures, real-time blockchain connection.

\section{Research Publications}

Theoretical foundations published on Zenodo with DOIs:

\begin{center}
\begin{longtable}{@{}lll@{}}
\toprule
\textbf{Paper} & \textbf{Topic} & \textbf{DOI} \\
\midrule
ERLHS & Hamiltonian framework for coherence-preserving ML & 10.5281/zenodo.17928909 \\
Karmonic Mesh & O(N log N) spectral consensus on toroidal manifolds & 10.5281/zenodo.17928991 \\
Proof of Coherence & QKD-based distributed consensus & 10.5281/zenodo.17929054 \\
Toroidal Mesh & 10K TPS with SPHINCS+ via parallel verification & 10.5281/zenodo.17931222 \\
Toroidal Governance & Tonnetz manifold governance & 10.5281/zenodo.17929091 \\
Augmented Democracy & Coherence-constrained democratic infrastructure & Preprint \\
\bottomrule
\end{longtable}
\end{center}

\section{References}

\begin{enumerate}
  \item NIST Post-Quantum Cryptography Standardization (2024)
  \item NIST FIPS 205: SPHINCS+
  \item NIST FIPS 206: Falcon
  \item NIST FIPS 203: ML-KEM
  \item Substrate Developer Documentation
  \item ETSI GS QKD 014: QKD Key Delivery API
  \item Grover, L. ``A Fast Quantum Mechanical Algorithm for Database Search'' (1996)
  \item Shor, P. ``Algorithms for Quantum Computation'' (1994)
\end{enumerate}

\section*{Contact}

\textbf{Project}: QuantumHarmony (QuantumVerse Protocols) \\
\textbf{Technical Lead}: Sylvain Cormier \\
\textbf{Repository}: \url{https://github.com/QuantumVerseProtocols/quantumharmony}

\vfill
\noindent
\textit{This document describes the system as implemented. No forward-looking claims are made.}

\end{document}
